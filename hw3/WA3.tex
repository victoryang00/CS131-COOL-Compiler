\documentclass[10pt]{article}
\usepackage[pdftex]{graphicx, color}
\usepackage{listings,proof}

\usepackage{tikz}
\usetikzlibrary{automata,positioning}

\headheight 8pt \headsep 20pt \footskip 30pt
\textheight 9in \textwidth 6.5in
\oddsidemargin 0in \evensidemargin 0in
\topmargin -.35in

\lstset{basicstyle=\small\ttfamily,breaklines=true}
\newcommand {\pts}[1]{{\bf #1 pts}}
\newcommand{\ttmath}[1]{$\mathtt{#1}$}
\newcommand{\ossimple}[6]{#1,#2,#3\vdash #4 : #5,#6}
\newcommand{\osrule}[8]{\frac{#7}{\ossimple{#1}{#2}{#3}{#4}{#5}{#6}}\eqno
  \mbox{#8}}
\newcommand{\infertext}[2]{\infer{{\textrm{#1}}}{#2}}

\begin{document}
\begin{center}
\Large CS131 Compilers: Writing Assignment 3\\Due Saturday, November 23, 2019 at 23:00 pm
\end{center}

\begin{center}
%% Change this:
\LARGE yangyw - 2018533218
\end{center}

This assignment asks you to prepare written answers to questions on
semantic analysis. Each of the questions has a short answer. You
may discuss this assignment with other students and work on the problems
together. However, your write-up should be your own individual work.
and you should indicate in your submission who you worked with, if applicable.
Written assignments are turned in at the start of lecture.
You should use the Latex template provided at the course web site to write your solution.

\begin{center}
%% Change this:
I worked with: Nobody
\end{center}

Example for type rule in tex:
\[\infertext
          {$O$[Bo/x][Ob/x] $\vdash$ x: Ob}
          {\textrm{$O$[Bo/x][Ob/x](x) = Ob}}       
\]

\begin{enumerate}
\item (10 pts) Consider the following class definitions.
\begin{verbatim}
  class A {
    i: Int
    o: Object
    b: B <- new B
    x: SELF_TYPE
    f(): SELF_TYPE {x}
  }
  class B inherits A {
    g(b: Bool): Object { (* EXPRESSION *) }
  }
\end{verbatim}

Ar eassume that the type checker implements the rules described in the lectures and in the Cool Reference
Manual. Foch of the following expressions, occurring in place of (* EXPRESSION *) in the body
of the method \emph{g}, show the static type inferred by the type checker for the expression. If the expression
causes a type error, give a brief explanation of why the appropriate type checking rule for the expression
cannot be applied.

\begin{enumerate}
  \item \begin{verbatim} x \end{verbatim}
  \item \begin{verbatim} self = x \end{verbatim}
  \item \begin{verbatim} self = i \end{verbatim}
  \item \begin{verbatim} let x: B <- x in x \end{verbatim}
  \item \begin{verbatim}
case o of
    o: Int => b;
    o: Bool => o;
    o: Object => true;
esac
\end{verbatim}
\end{enumerate}
\begin{enumerate}
\item \begin{verbatim}SELF_TYPE_{B} \end{verbatim}
\item \begin{verbatim}bool \end{verbatim}
\item \begin{verbatim}Error; because Int can only campared with Int, self is not Int.\end{verbatim}
\item \begin{verbatim}B\end{verbatim}
\item \begin{verbatim}bool\end{verbatim}
\end{enumerate}
\medskip

\item (10 pts) Show the full type \emph{derivation tree }for the following judgement:
(You can use Bo as the type Bool and Ob as the type Object)

\begin{center}
$O$[Bool/x] $\vdash$ \tt{x $<$- (let x:Object $<$- x in x = x): Bool}
\end{center}

\[ \infertext{$O$[Bo/x] $\vdash$ x: Bo  Bo $\leq$ Bo} {\textrm{$O$[Bo/x](x) = Bo}}
\]

\begin{equation}
  \infertext
    {$O$[Bo/x] $\vdash$ x: (let x:Ob $<$- x in x = x): Bo}
		{\textrm{\infertext{$O$[Bo/x] $\vdash$ x: Bo  Bo $\leq$ Bo}{\textrm{$O$[Bo/x](x) = Bo}}\ \ \infertext{O[Bo/x] $\vdash$ let x:Ob $<$- x in x = x: Bo}{{\infertext{$O$[Bo/x] $\vdash$ x: Ob Ob $<$= Ob}{\textrm{$O$[Bo/x](x) = Bo Bo $<$= Ob}}}{\infertext{$O$[Bo/x][Ob/x] $\vdash$ x = x: Ob}{\infertext{$O$[Bo/x][Ob/x] $\vdash$ x: Ob}{\textrm{$O$[Bo/x][Ob/x](x) = Ob}}\ \ \infertext{$O$[Bo/x][Ob/x] $\vdash$ x: Ob}{\textrm{$O$[Bo/x][Ob/x](x) = Ob}}}}}}}
\end{equation}

\medskip
\item (10 pts) Suppose we extend the grammar for Cool with a ``{\bf void}'' keyword
\begin{eqnarray*}
  expr & ::= & {\bf void} \\
       & |   & ...
\end{eqnarray*}

  that is analogous to {\tt null} in Java. (Currently objects are
  initialized to void if they have no other initializer specified, but
  there is no general-purpose {\bf void} keyword.)  We want to be able
  to use {\bf void} wherever an object can be used, as in
\begin{verbatim}
  let foo:Int <- if some_test
                 then 5
                 else void
                 fi
  in ...
\end{verbatim}

  Give a sound typing rule that we can add to the Cool specification
  to accommodate this new keyword.

  \[\infertext{
    \textrm{ $O, M, C \vdash$ void  : T }	} 
  {}\]

\medskip
\item  (10 pts) Suppose we extend Cool with exceptions by adding two new constructs
to the Cool language.
\begin{eqnarray*}
  expr & ::= & \textit{\bf try } expr \;\textit{\bf catch } ID => expr \\
       & |   & \textit{\bf throw } expr \\
       & | & ...
\end{eqnarray*}
Here {\bf try}, {\bf catch} and {\bf throw} are three new terminals.
  ``${\bf throw }\;\; expr$'' returns $expr$ to the
  closest dynamically enclosing catch block.
Note that since {\bf throw} expression returns control to a different location, we do not really
  care about the context in which throw is used. For example,
({\bf throw} $false) + 2$ is a valid Cool expression (However, note that
  ({\bf throw} $false) + (2+true)$ is not a valid Cool expression).  Following is an example that uses the
try-catch and throw constructs.
\begin{verbatim}
try
   if some_test1 then throw 34
   else if some_test2 then throw ``undefined error''
   else do_something fi fi
catch x =>
  case x of
      x:Int => do_something1
      x:String => do_something2
  esac
\end{verbatim}

The above program fragment executes
  ``do\_something1'' (with $x$ bound to the value 34) if ``some\_test1''
  evaluates to \textsf{true}. It executes ``do\_something2'' (with $x$ bound to the
  value ``undefined error'') if ``some\_test1'' evaluates to \textsf{false} but
  ``some\_test2'' evaluates to \textsf{true}. It executes ``do\_something'' if both
  ``some\_test1'' and ``some\_test2'' evaluate to \textsf{false}.


Give a set of new sound typing rules that we can add to the Cool specification
to accommodate these two new constructs.

\[\infertext
      {$O,M,C$ $\vdash$ throw $e$ : $T_2$ }
      {\textrm{$O,M,C$ $\vdash$ $e$ : $T_1$}
      }
    \]


\[\infertext
      {$O,M,C$ $\vdash$ \textbf{try} $e_1$ \textbf{catch} x $=>$ $e_2$ : $T_1\sqcup T_2$}
      {\textrm{$O,M,C$ $\vdash$ $e_1$ : $T_1$\ \ \ \ \ $O[Object/Id],M,C$ $\vdash$ $e_2$ : $T_2$}
      }
    \]



\medskip
\item (10 pts) The Java programming language includes arrays.  The Java
language specification states that if $s$ is an array of elements of
class $S$, and $t$ is an array of elements of class $T$, then the
assignment $s = t$ is allowed as long as $T$ is a subclass of $S$.
This typing rule for array assignments turns out to be unsound. (Java
works around the fact that this rule is not statically sound by inserting
runtime checks to generate an exception if arrays are used
unsafely. For this question, assume there are no special runtime checks.)

Consider the following Java program, which type checks according
to the proceeding rule:
\begin{verbatim}
class Mammal { String name; }

class Dog extends Mammal { void beginBarking() { ... } }

class Main {
    static public void main(String argv[]) {
        Dog x[] = new Dog[5];
        Mammal y[] = x;

        /* Insert code here */
    }
}
\end{verbatim}
Add code to the \texttt{main} method so that the resulting program is
a valid Java program (i.e., it type checks statically and so it will
compile), but the program could result in an error being applied
to an inappropriate type when executed.  Include a brief explanation
of how your program exhibits the problem.
\begin{verbatim}
    Dog x[] = new Dog[5];
    Mammal y[] = x;

    Mammal a_cat = new Mammal();
    y[0] = a_cat; // xxx
    x[0].beginBarking();
\end{verbatim}
\    The problem here is that arrays are not just lists of values -- they represent memory
locations into which we can store new data.\\
\ \    Normally we say that A $\leq$ B if objects of type A can safely be used anywhere that
objects of type B can be used. That’s not quite true with Dog[\ ] $\leq$ Mammal[\ ], since
a Mammal[\ ] object can be safely used on the left-hand side of the assignment marked by
xxx, while a Dog[\ ] object cannot. Java handles this by adding a runtime check to every
array assignment that determines whether the right-hand side of the assignment matches
the dynamic type of the array.
\medskip
\item (10 pts) Now that you know why Java arrays are problematic, you decide to add an array construct
to Cool with sound typing rules. An array containing objects of type A is declared as being of type
\textsf{Array(A)} and one can create arrays in Cool using the \textsf{new Array[A][e]} construct, where \textsf{e} is an
expression of type \textsf{Int}, specifying the size of the array.
One can access elements in the array using
the construct \textsf{e1[e2]} which yields the \textsf{e2}'th element in array \textsf{e1},
and one can insert elements into the array using the notation \textsf{e1[e2] $<$- e3}.
Finally, as in Java, an assignment from one array \textsf{a} to an
array \textsf{b} does not make copies of the elements contained in \textsf{a}, but addresses of elements.

\begin{enumerate}
  \item (2 pts) Give a sound subtype relation for arrays in Cool, i.e., state the conditions under which
the subtype relation \textsf{Array(T)$\le$ T'} is valid.
  \item Give sound typing rules that are as permissive as possible for the following constructs:

\begin{enumerate}
    \item (2 pts) \textsf{new Array[A][e]}
    \item (2 pts) \textsf{e1[e2]}
    \item (4 pts) \textsf{e1[e2] $<$- e3}. Assume the type of the whole expression is the type of \textsf{e1}.

    \[\infertext
    {$O,M,C$ $\vdash$ $e_1[e_2]$ $<-$  $e_3$ : $Array(T)$ }
    {\textrm{$O,M,C$ $\vdash$ $e_1$ : $Array(T)$}
    \ \ \ \ 	\textrm{$O,M,C$ $\vdash$ $e_2$ : $Int$}
    \ \ \ \ 	\textrm{$O,M,C$ $\vdash$ $e_3$ : $T'$}
    \ \ \ \   \textrm{T' $<=$ T}
    }
  \]
\end{enumerate}
\end{enumerate}
\end{enumerate}
\end{document}

